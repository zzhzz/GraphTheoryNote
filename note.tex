% !TEX program = xelatex
\documentclass[12pt, UTF8]{ctexart}
\usepackage{amsmath}
\begin{document}

\section{四色猜想的证明}

\subsection{Kempe在1879年给出的证明}
对极大平面图G的顶点数n归纳证明。\\
$n \leq 4$时,四色猜想成立;\\
设$n \ge 4$时,四色猜想成立,考察n的情况。\\
由欧拉公式可知,对于任意极大平面图G,有$3 \leq \delta \leq 5$成立。故按$\delta = 3,4,5$分类证明。
\begin{itemize}
\item $\delta = 3$,设 $x \in V(G), d(x)=3, N(x)=\{v_1,v_2,v_3\}$。由归纳假设,$\exists f \in C^0_4(G-x), f(N(x)) = {1, 2, 3}$,由此得到G的一个4-着色$f'$:对于$\forall u \in V(G),$
\[
    f'(u) = \begin{cases}
        4, u = x \\
        f(u), otherwise 
    \end{cases}
\]
\item 当$\delta=4$时,设$x \in V(G), d(x) = 4, N(x) = \{v_1, v_2, v_3, v_4\}$,由归纳假设,$\exists f \in C^0_4(G-x)$,使得$|f(N(x))|=2,3,4$。当$|f(N(x))|=2,3$时,类似前者,可得到G的一个4-着色。故只考虑$|f(N(x))|=4$的情况。不妨设$f(v_i)=i, i=1,2,3,4$,在导出子图$G_{13}$中,若顶点$v_1$与$v_3$不在同一个连通分支,将顶点$v_1$所在的13-分支颜色互换,其他顶点颜色不变,得到G的一个4-着色$f'$:对于$\forall u \in V(G),$
\[
    f'(u) = \begin{cases}
        1, u = x\ or\ u \in V(G_{13}^{v_1}), f(u) = 3\\
        3, u \in V(G_{13}^{v_1}), f(u) = 1\\
        f(u), otherwise 
    \end{cases}
\]
其中$G_{13}^{v_1}$表示着色$f$下顶点$v_1$所在的13-分支。\\
故假设在$G_{13}$中,顶点$v_1$与$v_3$在同一个连通分支,则顶点$v_2$和顶点$v_4$不在同一个24-分支,将顶点$v_2$所在的24-分支颜色互换,其他顶点颜色不变,得到G的一个4-着色$f'$:对于$\forall u \in V(G),$
\[
    f'(u) = \begin{cases}
        2, u = x\ or\ u \in V(G_{24}^{v_2}), f(u) = 4\\
        4, u \in V(G_{24}^{v_2}), f(u) = 2\\
        f(u), otherwise
    \end{cases}
\]
故$\delta = 4$时,结论成立。

\item 当$\delta=5$时,设$x \in V(G), d(x) = 5, N(x) = \{v_1, v_2, v_3, v_4, v_5\}$,由归纳假设,$\exists f \in C^0_4(G-x)$,使得$|f(N(x))|=3,4$。当$|f(N(x))|=3$时,类似前者,可得到G的一个4-着色。故只考虑$|f(N(x))|=4$的情况。不妨设$f(v_i)=i, i=1,2,3,4, f(v_5)=2$。在导出子图$G_{13}$中,若顶点$v_1$与$v_3$不在同一个连通分支(或在$G_{14}$中,顶点$v_1$与顶点$v_4$不在同一个连通分支),则将顶点$v_1$所在的13(或14)-分支颜色互换,其他顶点颜色不变,得到G的一个4-着色$f'$:对于$\forall u \in V(G),$
\[
    f'(u) = \begin{cases}
        1, u = x\ or\ u \ in V(G_{13}^{v_1})(V(G_{14}^{v_1})), f(u) = 3(4)\\
        3(4), u \in V(G_{13}^{v_1})(V(G_{14}^{v_1})), f(u) = 1\\
        f(u), otherwise
    \end{cases}
\]
从而结论成立。故设在$G_{13}$中,$v_1$与$v_3$在同一个连通分支,且在$G_{14}$中,$v_1$与$v_4$在同一个连通分支,则$v_2$与$v_4$不在同一个24-分支,且$v_3$与$v_5$不在同一个23-分支。将顶点$v_2$所在的24-分支颜色互换,同时将顶点$v_5$所在的23-分支颜色互换,其他顶点颜色不变,可得到G的一个4-着色$f'$:对于$\forall u \in V(G),$
\[
    f'(u) = \begin{cases}
        2, u = x\ or u \in V(G_{24}^{v_2}),f(u)=4\ or\ u \in V(G_{23}^{v_5}),f(u)=3\\
        3, u \in V(G_{23}^{v_5}), f(u) = 2\\
        4, u \in V(G_{24}^{v_2}), f(u) = 2\\
        f(u), otherwise
    \end{cases}
\]
故$\delta = 5$时,结论成立。
\end{itemize}

\subsection{Kempe证明的缺陷}
1890年,Heawood发现Kempe证明过程中的缺陷:当$\delta = 5$时,若$v_1$与$v_3$所在的13-连通分支与$v_1$与$v_4$所在的14-连通分支相交于颜色1的顶点数$\ge 2$时,则Kempe的证明出现错误:将$v_2$所在的24-分支颜色互换后,无法确定$v_3$与$v_5$在$G_{23}$中是否连通,因此也就无法进行第二次换色。Heawood与Kempe都无法修正这个缺陷,但利用Kempe“证明"四色猜想的过程中,很容易得到”五色定理“。

\subsection{Kempe变换}
基于Kempe的方法,很容易从图的一种着色到处另一种着色,后人称之为Kempe变换。利用Kempe变换可以大大降低求一个图所有着色的复杂度。

\subsection{同阶极大平面图的构造}
1936年,Wagner提出了边翻转算子的概念。设G是一个极大平面图,abcd是G中以ac为腰的菱形。若在该菱形中删除边ac后添加边bd,使所得的极大平面图仍是极大平面图,则称此运算为G中对ac的边翻转运算,并将ac成为可翻转的。显然,如果G本身含边bd,则对ac不能实施边翻转运算,即ac不是可翻转的。\\
定理:任意n($> 4$)-阶极大平面图G至少包含n-2条可翻转边,并且存在一类阶数为n=3t-4的极大平面图,恰好包含n-2条可翻转边,其中$t \ge 3$.\\


\subsection{异阶极大平面图的构造}
纯弦圈的定义:设G是一个极大平面图,C是G中的一个圈,若圈C内不含顶点,且C的每个面都是三角形,则把圈C成为图G的一个纯弦圈,并把C内每条边称为圈C的弦。极大平面图中的三角形也是为纯弦圈。\\
1891年,Eberhard开展了对极大平面图构造问题的研究,给出了能够构造所有极大平面图的运算系统,并把这个运算系统记为$\langle K_4;\Phi=\{\phi_1,\phi_2,\phi_3\}\rangle$,其中$K_4$表示初始对象,$\Phi$为运算集,$\phi_1,\phi_2,\phi_3$是三种算子。\\
定理:任意n($\ge 4$)-阶极大平面图G可通过对(n-1)-阶极大平面图实施运算$\phi_1,\phi_2,\phi_3$得到

\subsection{极大平面图的生成运算系统}
扩2-轮运算步骤:
\begin{itemize}
    \item 在某条边uv的两端点之间再连接一条边,使其产生2重边,即产生2-圈;
    \item 在该2-圈内部添加一个新的顶点x,并令x与2-圈的两顶点u与v相连边,产生一个2-轮。
\end{itemize}
缩2-轮运算步骤的作用对象子图是一个2-轮,步骤为:
\begin{itemize}
    \item 删除该2-轮的轮心及相关联的两条边
    \item 删除2重边中的一条
\end{itemize}
扩3-轮运算:在极大平面图的某一个面上加入一个顶点x,并让x与构成该面的3个顶点相连边。因此,扩3-轮运算在极大平面图中的对象子图是一个三角形。\\
缩3-轮运算:将某个3度顶点及与该顶点相关联的边删去。\\
扩4-轮运算的步骤:
\begin{itemize}
    \item 在极大平面图中某条2-长路$P_3=v_1v_2v_3$上,从顶点$v_1$出发,沿着$v_1 \rightarrow v_2 \rightarrow v_3$方向,从边-点-边的内部划开,即将边$v_1v_2$,顶点$v_2$及边$v_2v_3$从中间划开,使得顶点$v_2$变成两个顶点,分别记作$v_2$与$v_2'$;$v_1v_2$与$v_2v_3$均变成了两条边,分别是$v_1v_2$ 和$v_1v_2'$, $v_2v_3$和$v_2'v_3$,原来在$P_3$左侧与$v_2$关联的边变成了与$v_2$关联,原来在$P_3$右侧与$v_2$关联的边变成与$v_2'$关联,从而保持平面性。
    \item 在顶点$v_1,v_2',v_3,v_4$这4个顶点形成的4-圈内增加一个新顶点x,并将x分别与其余4个顶点连接。
\end{itemize}
缩4-轮运算的步骤:在极大平面图中,将某4度顶点以及与它关联的边均删去,并且对该顶点领域中的某一对不相邻的顶点实施收缩运算。\\
扩5-轮运算的步骤:
\begin{itemize}
    \item 对极大平面图中某漏斗子图$L=v_1-\Delta v_2v_3v_4$,从顶点$v_1$出发,沿着$v_1 \rightarrow v_2$方向,从边-点内部划开,即将边$v_1v_2$,顶点$v_2$从中间划开,使得顶点$v_2$变成两个顶点,分别记作$v_2$与$v_2'$;$v_1v_2$变成了两条边,分别是$v_1v_2$与$v_1v_2'$;原来在$L$左侧与$v_2$关联的边变成与$v_2$关联,原来在$L$右侧与$v_2$关联的边变成与$v_2'$关联,从而保持平面性。
    \item 在顶点$v_1,v_2',v_3,v_4,v_2$这5个顶点形成的5-圈内增加一个新顶点x,并将x分别与顶点$v_1,v_2',v_3,v_4,v_2$相连边。
\end{itemize}
缩5-轮运算的步骤:在极大平面图中,将某5度顶点以及与它关联的边均删去,并且对该顶点领域中的某一对不相邻的顶点实施收缩运算。\\

定理:设G是一个n-阶极大平面图,则可通过不断地试试缩2-轮,缩3-轮,缩4-轮或缩5-轮运算,使得改图最终收缩成$K_3$。

\subsection{纯树着色}
猜想:若G是一个唯一3-边可着色立方图,则G是平面图且汗一个三角形\\
GK-猜想:一个平面图G是唯一4-可着色的充分必要条件是G为地柜极大平面图。\\
设G是一个4-色极大平面图,颜色集$C(4)=\{1,2,3,4\}, f \in C_4^0(G)$。若G中友谊长度为2m的偶圈$C_{2m}, V(C_{2m})={v_1,v_2,...,v_{2m}}$,使得$\{f(v_1), f(v_2),...,f(v_{2m})\}$中只含有2种颜色,则称$C_{2m}$是$f$的一个2-色圈,也成为$f$含有2-色圈,并称$f$为圈着色,称G是可圈着色的。若$C_{2m}$上所含颜色为i和t,则$C_{2m}$亦可记作it-圈。否则,若$f$不含2-色圈,则称$f$为图$G$的树着色,称G时可树着色的。在圈着色与树着色分类的基础上,相应地可将G分为3种类型:纯树着色型,即$C_4^0(G)$每个着色均为树着色;纯圈着色型,即$C_4^0(G)$中每个着色均为圈着色;混合着色型,即$C_4^0(G)$中即含树着色,又含圈着色。


\subsection{Kempe变换与Kempe等价类}
Kempe-等价:令$f,f' \in C_4^0(G)$,若从$f$出发,通过若干次Kempe变换可获得$f'$,则称$f$与$f'$是K-等价的\\
Kempe图:设G是一个k-可着色图,若G的所有k-着色是K-等价的,则称G为k-Kempe图\\
Kempe等价类:所有与$f$互为K-等价的着色与$f$的并构成之集,记作$F^f(G)$\\
2-色圈:设G是一个4-色极大平面图,若C是G的一个偶圈,$|f(C)|=2$,则称C是f的2-色圈。C上的两种颜色称为圈色。\\

树型Kempe等价类:若f是G的一个树着色,则$C^2(f)=\phi$,即f中全部6个2-色导出子图是连通的,因此,$F^f(G)=\{f\}$。我们把这种非4-Kempe图G的Kempe等价类成为树型Kempe等价类,并把G成为树型极大平面图。\\
定理:设G是一个非唯一4-色的可树着色极大平面图,则$K(G,4) \ge 2$;若G时纯树着色的,则$K(G,4)=|C^0_4(G)|$;若G是混合型着色,且含树着色的数目为t个,则$K(G,4) \ge t+1$。

2-色不变圈:一个Kempe等价类中全部为2-色圈\\

\end{document}